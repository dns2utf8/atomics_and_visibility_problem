\documentclass[10pt]{beamer}

\usetheme[progressbar=frametitle]{metropolis}
\usepackage{appendixnumberbeamer}

% Syntax highlighting
\usepackage{minted}

\usepackage{booktabs}
\usepackage[scale=2]{ccicons}

\usepackage{pgfplots}
\usepgfplotslibrary{dateplot}

\usepackage{xspace}
\newcommand{\themename}{\textbf{\textsc{metropolis}}\xspace}

\title{Atomics and visibility problem}
\subtitle{Safe concurrent operations Integer operations}
% \date{\today}
\date{Wednesday, 24. October 2018}
\author{Stefan Schindler @dns2utf8}
\institute{\href{https://www.meetup.com/de-DE/Rust-Zurich/}{Rust Zürichsee Meetup} hosted by \href{https://coredump.ch}{coredump.ch}}
\titlegraphic{\hfill\includegraphics[height=1.5cm]{rust-logo-blk.png}}

%gets rid of bottom navigation bars
\setbeamertemplate{footline}[page number]{}

%gets rid of navigation symbols
%\setbeamertemplate{navigation symbols}{}

\begin{document}

\maketitle

\begin{frame}{Table of contents}
  \setbeamertemplate{section in toc}[sections numbered]
  \tableofcontents[hideallsubsections]
\end{frame}

\section{Introduction}

\begin{frame}[fragile]{Who I am?}

My name is Stefan and I ...

\begin{itemize}
\item studied Computer Science at HSR
\item started with Rust in 2016
\item maintain crates: threadpool, wipe\_buddy, son\_grid\_engine, ... \href{https://crates.io/users/dns2utf8}{some more}
\item organize \href{https://rustfest.eu}{RustFest.eu} next up in Rome November 24th to 27th
\item talk about rust
\item am looking for a new rusty job
\end{itemize}

\end{frame}
\begin{frame}{What will we learn tonight?}
  
\begin{itemize}
\item What is a Symbol?
\item What is an Atomic Operation?
\item What is the visibility problem?
\item How to solve it?
\end{itemize}
\end{frame}

\section{Integer Data}

\begin{frame}[fragile]{Where is index?}

\begin{minted}{rust}
let data = vec![42, 42, 42, 42];

let mut index = 0; // <-- what is the kind of index?
let length = data.len();
while index < length {
    println!("{}: {}", index, data[index]);
    index += 1
}
\end{minted}
and \alert{where} will it be stored at runtime?
\end{frame}

\section{Watching a value from another thread}

\begin{frame}{Scenario}
    \begin{itemize}
        \item Control thread allocates global memory for \texttt{threshold}
        \item $Thread_W$ will wait for \texttt{threshold} to pass a certain value and then alert the user
        \item $Thread_S$ is waiting over input from the world and updates \texttt{threshold}
    \end{itemize}
\end{frame}

\begin{frame}[fragile]{Monitor data thread - 0}
\begin{minted}{rust}
use std::thread::{sleep, spawn};
use std::time::Duration;
#[allow(non_upper_case_globals)]
static mut threshold: isize = 0;
const MAX_TEST: usize = 100000;
fn main() {
  let _counter = spawn(|| {
    loop {
// note: mutable statics can be mutated by multiple 
// threads: aliasing violations or data races will 
// cause undefined behavior
      unsafe {
          threshold = (threshold + 1) % 100;
          //println!("counter: {}", threshold);
      }
    }
  });
\end{minted}
\end{frame}

\begin{frame}[fragile]{Monitor data thread - 1}
\begin{minted}{rust}
let watcher = spawn(|| {
    sleep(Duration::from_millis(500));
    let mut history = Vec::with_capacity(MAX_TEST);
    let mut last = unsafe { threshold };
    let mut count = 0;
    for _ in 0..MAX_TEST {
        let threshold_local = unsafe { threshold };
        if last == threshold_local {
            count += 1;
        } else {
            history.push((last, count));
            last = threshold_local;
            count = 0;
        }
    }
    history
});

\end{minted}
\end{frame}

\begin{frame}[fragile]{Monitor data thread - 2}
\begin{minted}{rust}
// back in fn main() { ...
  let history = watcher.join().expect("watcher failed");
  println!("{:?}\nn transitions recorded: {}"
            , history, history.len());
  //_counter.join();
}

\end{minted}
What kind of out put would you \alert{expect}?

Would release mode change that output?
\end{frame}



\begin{frame}[fragile]{Monitor data thread - 3}
\metroset{block=fill}

      \begin{block}{Debug mode}
\begin{minted}{rust}
[ ...
    (88, 0), (93, 0), (98, 0), (4, 0), (10, 0),
    (15, 0), (20, 0), (26, 0), (31, 0), (36, 0),
    (41, 0), (46, 0), (53, 0), (58, 0), (63, 0)
]
n transitions recorded: 99769
\end{minted}
      \end{block}
\end{frame}
\begin{frame}[fragile]{Monitor data thread - 4}
\metroset{block=fill}

      \begin{block}{Debug mode}
\begin{minted}{rust}
[ ...
    (88, 0), (93, 0), (98, 0), (4, 0), (10, 0),
    (15, 0), (20, 0), (26, 0), (31, 0), (36, 0),
    (41, 0), (46, 0), (53, 0), (58, 0), (63, 0)
]
n transitions recorded: 99769
\end{minted}
      \end{block}

    \begin{block}{Release mode}
\begin{minted}{rust}
[]
n transitions recorded: 0
\end{minted}
      \end{block}
\end{frame}

\begin{frame}[fragile]{Manipulate data in RAM - 0}
What if we have a list of objects and we need the value furthest away from Zero?
\begin{minted}{rust}
fn absolute_max(result: &mut i64, list: &Vec<i64>) {
    for i in list {
        let abs = if i < 0 { -i } else { i };
        if result < abs {
            result = i;
        }
    }
}
\end{minted}
Quick question: \emph{Where is the data?} and why does it \alert{not} compile?
\end{frame}

\begin{frame}[fragile]{Manipulate data in RAM - 1}
Making it compile with rust reveals the problem:
\begin{minted}{rust}
fn absolute_max(result: &mut i64, list: &Vec<i64>) {
    for i in list {
        let i = *i;
        let result_local = *result;
        
        let abs_i = if i < 0 { -i } else { i };
        let abs_r = if result_local < 0 { -result_local } else { result_local };
        if abs_r < abs_i {
            *result = i;
        }
    }
}
\end{minted}
Full source: \url{https://play.rust-lang.org/?gist=11b541f0b4165f1cc39472c15f494a00&version=stable&mode=debug&edition=2015}
\end{frame}




\begin{frame}{Tables}
  \begin{table}
    \caption{Largest cities in the world (source: Wikipedia)}
    \begin{tabular}{lr}
      \toprule
      City & Population\\
      \midrule
      Mexico City & 20,116,842\\
      Shanghai & 19,210,000\\
      Peking & 15,796,450\\
      Istanbul & 14,160,467\\
      \bottomrule
    \end{tabular}
  \end{table}
\end{frame}
\begin{frame}{Blocks}
  Three different block environments are pre-defined and may be styled with an
  optional background color.

  \begin{columns}[T,onlytextwidth]
    \column{0.5\textwidth}
      \begin{block}{Default}
        Block content AAA.
      \end{block}

      \begin{alertblock}{Alert}
        Block content AAA.
      \end{alertblock}

      \begin{exampleblock}{Example}
        Block content AAA.
      \end{exampleblock}

    \column{0.5\textwidth}

      \metroset{block=fill}

      \begin{block}{Default}
        Block content.
      \end{block}

      \begin{alertblock}{Alert}
        Block content.
      \end{alertblock}

      \begin{exampleblock}{Example}
        Block content.
      \end{exampleblock}

  \end{columns}
\end{frame}

\begin{frame}{Math}
  \begin{equation*}
    e = \lim_{n\to \infty} \left(1 + \frac{1}{n}\right)^n
  \end{equation*}
\end{frame}

\begin{frame}{Line plots}
  \begin{figure}
    \begin{tikzpicture}
      \begin{axis}[
        mlineplot,
        width=0.9\textwidth,
        height=6cm,
      ]

        \addplot {sin(deg(x))};
        \addplot+[samples=100] {sin(deg(2*x))};

      \end{axis}
    \end{tikzpicture}
  \end{figure}
\end{frame}
\begin{frame}{Bar charts}
  \begin{figure}
    \begin{tikzpicture}
      \begin{axis}[
        mbarplot,
        xlabel={Foo},
        ylabel={Bar},
        width=0.9\textwidth,
        height=6cm,
      ]

      \addplot plot coordinates {(1, 20) (2, 25) (3, 22.4) (4, 12.4)};
      \addplot plot coordinates {(1, 18) (2, 24) (3, 23.5) (4, 13.2)};
      \addplot plot coordinates {(1, 10) (2, 19) (3, 25) (4, 15.2)};

      \legend{lorem, ipsum, dolor}

      \end{axis}
    \end{tikzpicture}
  \end{figure}
\end{frame}


{%
\setbeamertemplate{frame footer}{My custom footer}
\begin{frame}[fragile]{Frame footer}
    \themename defines a custom beamer template to add a text to the footer. It can be set via
    \begin{verbatim}\setbeamertemplate{frame footer}{My custom footer}\end{verbatim}
\end{frame}
}

%\begin{frame}{References}
%  Some references to showcase [allowframebreaks] \cite{knuth92,ConcreteMath,Simpson,Er01,greenwade93}
%\end{frame}

\section{Atomic Operations}

\begin{frame}{Performance differences}
    \cite{AMD64ArchVol1}
\end{frame}

\section{Conclusion}

\begin{frame}{Summary}

  Multi-Thread-Programms require atomic operations.
  Using them with an abstraction allows us to keep the development speed up.

  \begin{center}Slides: \url{https://gitlab.com/dns2utf8/rust_atomics}\end{center}

  %\begin{center}\ccbysa\end{center}

\end{frame}

{\setbeamercolor{palette primary}{fg=black, bg=yellow}
\begin{frame}[standout]
  Questions?
\end{frame}
}

\appendix

\begin{frame}[fragile]{Backup slides}
  Sometimes, it is useful to add slides at the end of your presentation to
  refer to during audience questions.

  The best way to do this is to include the \verb|appendixnumberbeamer|
  package in your preamble and call \verb|\appendix| before your backup slides.

  \themename will automatically turn off slide numbering and progress bars for
  slides in the appendix.
\end{frame}

\begin{frame}[allowframebreaks]{References}

  \bibliography{slides}
  \bibliographystyle{abbrv}

\end{frame}

\end{document}
